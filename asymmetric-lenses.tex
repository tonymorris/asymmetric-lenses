\documentclass{beamer}
\usepackage{amsmath}
\usepackage{cclicenses}
\usepackage[utf8]{inputenc}
\usepackage{graphics}
\usepackage{hyperref}
\usepackage{xcolor}
\usepackage{wasysym}
\usepackage{listings}
\usepackage{tikz}
\usepackage{amssymb}
\usepackage{textcomp}
\usepackage{verbatim}

\usetheme{Warsaw}
\usecolortheme{lily}
\setbeamercovered{transparent}
\setbeamertemplate{headline}{
  \begin{beamercolorbox}{section in head/foot}
    \vskip2pt\insertnavigation{\paperwidth}\vskip2pt
  \end{beamercolorbox}
}

\setbeamertemplate{footline}{
}

\begin{document}
\title{\large Asymmetric Lenses}

\author{
  {\small Tony Morris\\}
}
\date{{\footnotesize 24 July 2012, Brisbane Functional Programming Group}}

\begin{frame}
  \titlepage
\end{frame}

\begin{frame}
\centering
\emph{This Presentation is Licensed Under:}

Creative Commons Attribution-ShareAlike 3.0 Unported

\href{''http://creativecommons.org/licenses/by-sa/3.0/''}{CC BY-SA 3.0}

\cc \bysa

\end{frame}

\begin{frame}
\frametitle{Lens is CoState Comonad Coalgebra}

\begin{block}{It is true}
  \begin{itemize}
  \item A Lens is exactly the Coalgebra for the CoState Comonad\footnote{perhaps better known as the Store Comonad}\footnote{\emph{Russell O'Connor}}
  \item This is a concise, fully-describing statement, so we can go home now\ldots
  \item \ldots or we can examine the implications
  \end{itemize}
\end{block}
\end{frame}


\begin{frame}
\frametitle{The \lstinline$modify$ Problem}

\begin{block}{Record types}
\lstinputlisting[label=lst:Person.hs,caption=Person data type,language=haskell]{source/Person.hs}
\end{block}

\end{frame}

\begin{frame}
\frametitle{The \lstinline$modify$ Problem}

\begin{itemize}
\item Supposing we wish to reverse or upper-case a person's name
\item \lstinline[language=haskell]$modifyName :: (String -> String) -> Person -> Person$
\item We have to write this function for \emph{every} record field.
\end{itemize}

\end{frame}

\begin{frame}
\frametitle{It Gets Worse}

\begin{itemize}
\item Worse still, we would have to write many other functions and  per record field
  \begin{itemize}
  \item \lstinline[language=haskell]$setName :: String -> Person -> Person$
  \item \lstinline[language=haskell]$setNameAddress :: (String, Address) -> Person -> Person$
  \end{itemize}
\item Worse again, we would have to do this as record fields are embedded
\end{itemize}

\end{frame}


\begin{frame}[fragile]
\frametitle{Introducing Lenses}

\begin{block}{Lens is a Data Type}
\lstinputlisting[label=lst:Lens.hs,language=haskell]{source/Lens.hs}
\end{block}
\begin{block}{Example Lens}
\begin{lstlisting}[language=haskell]
nameLens :: Lens Person String
\end{lstlisting}
\end{block}

\end{frame}

\begin{frame}[fragile]
\frametitle{A universal \lstinline$modify$}

\begin{block}{A universal \lstinline$modify$}
\lstinputlisting[label=lst:modify.hs,language=haskell]{source/modify.hs}
\end{block}

\end{frame}


\begin{frame}[fragile]
\frametitle{Composing Lenses}

\begin{itemize}
\item We now have a universal \lstinline[language=haskell]$modify$ function to run with any lens
\item How do we handle embedded fields?
  \begin{itemize}
  \item A person has an address and an address has a suburb
  \item How can we get to the person's suburb?
  \end{itemize}
\end{itemize}

\end{frame}

\begin{frame}[fragile]
\frametitle{Composing Lenses}

\begin{block}{Compose the name and suburb lenses}
\begin{lstlisting}[language=haskell]
nameLens :: Lens Person Address
suburbLens :: Lens Address String
\end{lstlisting}
\end{block}

\begin{block}{Can we compose them?}
\begin{lstlisting}[language=haskell]
Lens Person Address
-> Lens Address String
-> Lens Person String
\end{lstlisting}
\end{block}

\end{frame}

\begin{frame}[fragile]
\frametitle{Lens is a Semigroupoid}

\begin{block}{What is a Semigroupoid?}
\begin{lstlisting}[language=haskell]
(>>>) ::
  semi a b
  -> semi b c
  -> semi a c
\end{lstlisting}

\end{block}

\begin{block}{Example Semigroupoids}
\begin{itemize}
\item \lstinline$(->)$
\item \lstinline$Monad m => Kleisli m$
\item \lstinline$Lens$
\end{itemize}
\end{block}

\end{frame}

\begin{frame}[fragile]
\frametitle{Traversing the Graph}

\begin{block}{Since Lens is a Semigroupoid \ldots}
\begin{itemize}
\item We can compose to an arbitrary depth
\item To reverse a person's suburb
\begin{lstlisting}[language=haskell]
reverseSuburb :: Person -> Person
reverseSuburb =
  modify 
    (addressLens >>> suburbLens)
    reverse
\end{lstlisting}
\item No more code shaped like a Greater Than symbol!

\end{itemize}
\end{block}

\end{frame}


\begin{frame}[fragile]
\frametitle{Lens Libraries}

% (|||) :: Lens a x -> Lens b x -> Lens (Either a b) x
% (***) :: Lens a b -> Lens c d -> Lens (a, c) (b, d)
% factor :: Lens (Either a b) (Either a c) -> Lens a (Either b c)
% distribute :: Lens a (Either b c) -> Lens (Either a b) (Either a c)
% unzip :: Lens s (a, b) -> (Lens s a, Lens s b)
% identity :: Lens a a

Not only does a Lens give rise to a Semigroupoid, but it also maps a set onto itself

\begin{block}{Identity Lens}
\begin{lstlisting}[language=haskell]
identityLens :: Lens a a
identityLens = Lens (const id) id
\end{lstlisting}
\end{block}

A Semigroupoid that has an identity element is called a \emph{Category}. Lens is a category.

\end{frame}

\begin{frame}[fragile]
\frametitle{Lens Libraries}

Two split lenses that map onto a value of the same element type produce a lens that can merge.

\begin{block}{Split Lens}
\begin{lstlisting}[language=haskell]
(|||) ::
  Lens a x
  -> Lens b x
  -> Lens (Either a b) x
Lens s1 g1 ||| Lens s2 g2 =
  Lens (either
          (\a -> Left . s1 a) 
          (\b -> Right . s2 b))
       (either g1 g2)
\end{lstlisting}
\end{block}

\end{frame}

\begin{frame}[fragile]
\frametitle{Lens Libraries}

Two disjoint lenses can be paired.

\begin{block}{Lens Tensor}
\begin{lstlisting}[language=haskell]
(***) ::
  Lens a b
  -> Lens c d
  -> Lens (a, c) (b, d)
Lens s1 g1 *** Lens s2 g2 =
  Lens (\(a, c) (b, d) -> (s1 a b, s2 c d))
       (\(a, c) -> (g1 a, g2 c))
\end{lstlisting}
\end{block}

\end{frame}

\begin{frame}[fragile]
\frametitle{Lens Libraries}

\begin{block}{And More}
\begin{itemize}
\item \begin{lstlisting}[language=haskell]
unzip ::
  Lens s (a, b)
  -> (Lens s a, Lens s b)
\end{lstlisting}

\item \begin{lstlisting}[language=haskell]
factor ::
  Lens (Either (a, b) (a, c)) 
       (a, Either b c)
\end{lstlisting}

\item \begin{lstlisting}[language=haskell]
distribute ::
  Lens (a, Either b c)
       (Either (a, b) (a, c))
\end{lstlisting}
\end{itemize}
\end{block}

\end{frame}

\begin{frame}[fragile]
\frametitle{Lens Values}

\begin{block}{First Lens}
\begin{lstlisting}[language=haskell]
fstLens :: Lens (a, b) a
fstLens =
  Lens (\(_, b) a -> (a, b)) fst
\end{lstlisting}
\end{block}
\begin{block}{Second Lens}
\begin{lstlisting}[language=haskell]
sndLens :: Lens (a, b) b
sndLens =
  Lens (\(a, _) b -> (a, b)) snd
\end{lstlisting}
\end{block}

\end{frame}

\begin{frame}[fragile]
\frametitle{Lens Values}

\begin{block}{Lenses on collections}
\begin{itemize}
\item \begin{lstlisting}[language=haskell]
mapLens ::
  Ord k => k -> Lens (Map k v) (Maybe v)
\end{lstlisting}

\item \begin{lstlisting}[language=haskell]
setLens ::
  Ord a => a -> Lens (Set a) Bool
\end{lstlisting}
\end{itemize}
\end{block}

\end{frame}


\begin{frame}
\frametitle{Partial Lenses}

\begin{block}{Sum types}
\begin{itemize}
\item Partial lenses provide \emph{nullability} through the lens
\item As a regular lens corresponds to fields of a record type, a partial lens corresponds to constructors of a sum type
\item \lstinputlisting[label=lst:Lens.hs,language=haskell]{source/PartialLens.hs}
\end{itemize}
\end{block}

\end{frame}

\begin{frame}
\frametitle{Example Sum Type}

\begin{block}{JSON Data Type}
\lstinputlisting[label=lst:Lens.hs,language=haskell]{source/Json.hs}
How do we navigate this data structure to perform retrieval and updates?
\end{block}

\end{frame}

\begin{frame}[fragile]
\frametitle{Like this}

\begin{block}{With a gigantic \textgreater symbol that's how}
\begin{lstlisting}[language=haskell]
case j of
  JArray x -> 
    case x of
      [] -> j
      (h:t) -> JArray (case h of 
        JArray y ->
          case y of
            a:b:u -> JArray (b:a:u)
            _ -> j) t
            _ -> j
  _ -> j
\end{lstlisting}
\end{block}

\end{frame}

\begin{frame}[fragile]
\frametitle{I Kid}

\begin{block}{Partial Lenses:}
\begin{itemize}
\item compose as their own semigroupoid
\begin{lstlisting}[language=haskell]
(>>>) ::
  PartialLens a b ->
  PartialLens b c ->
  PartialLens a c
\end{lstlisting}
\item split and merge
\begin{lstlisting}[language=haskell]
(|||) ::
  PartialLens a x
  -> PartialLens b x
  -> PartialLens (Either a b) x
\end{lstlisting}
\item \ldots and all those other helpful bits too!
\end{itemize}
\end{block}

\end{frame}


\begin{frame}
\frametitle{Language Support}

\begin{itemize}
\item All this boilerplate generating lenses for fields and constructors is a small price
\item But do we have to pay it?
\item Can the language do it for us?
\item We want to generate values with type \lstinline$a `Lens` b$ \ldots
\item \ldots instead of \lstinline$a -> b$ as in Haskell, Scala and everyone else
\end{itemize}

\end{frame}

\begin{frame}
\frametitle{Language Support}

\begin{itemize}
\item Boomerang \textemdash A bidirectional programming language for ad-hoc, textual data. 
\item Roy Programming Language \textemdash Brian McKenna (TBD)
\item Template Haskell
\item \emph{Your Programming Language}
\end{itemize}

\end{frame}


\begin{frame}
\frametitle{Further Topics}

\begin{itemize}
\item Lenses with Polymorphic Update
\item Fusing Lenses on the target to the pair of set/get (Store)
\item Optimal Integration into a General Purpose Programming Language
\item Lenses must obey laws
\end{itemize}

\end{frame}


\appendix
\lstinputlisting[label=lst:Lens.hs,language=haskell]{source/Full.hs}

\end{document}

