\begin{frame}[fragile]
\frametitle{Lens Libraries}

% (|||) :: Lens a x -> Lens b x -> Lens (Either a b) x
% (***) :: Lens a b -> Lens c d -> Lens (a, c) (b, d)
% factor :: Lens (Either a b) (Either a c) -> Lens a (Either b c)
% distribute :: Lens a (Either b c) -> Lens (Either a b) (Either a c)
% unzip :: Lens s (a, b) -> (Lens s a, Lens s b)
% identity :: Lens a a

Not only does a Lens give rise to a Semigroupoid, but it also maps a set onto itself

\begin{block}{Identity Lens}
\begin{lstlisting}[language=haskell]
identityLens :: Lens a a
identityLens = Lens (const id) id
\end{lstlisting}
\end{block}

A Semigroupoid that has an identity element is called a \emph{Category}. Lens is a category.

\end{frame}

\begin{frame}[fragile]
\frametitle{Lens Libraries}

Two split lenses that map onto a value of the same element type produce a lens that can merge.

\begin{block}{Split Lens}
\begin{lstlisting}[language=haskell]
(|||) ::
  Lens a x
  -> Lens b x
  -> Lens (Either a b) x
Lens s1 g1 ||| Lens s2 g2 =
  Lens (either
          (\a -> Left . s1 a) 
          (\b -> Right . s2 b))
       (either g1 g2)
\end{lstlisting}
\end{block}

\end{frame}

\begin{frame}[fragile]
\frametitle{Lens Libraries}

Two disjoint lenses can be paired.

\begin{block}{Lens Tensor}
\begin{lstlisting}[language=haskell]
(***) ::
  Lens a b
  -> Lens c d
  -> Lens (a, c) (b, d)
Lens s1 g1 *** Lens s2 g2 =
  Lens (\(a, c) (b, d) -> (s1 a b, s2 c d))
       (\(a, c) -> (g1 a, g2 c))
\end{lstlisting}
\end{block}

\end{frame}

